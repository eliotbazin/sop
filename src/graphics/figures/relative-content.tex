% read in data
\pgfplotstableread{
	AA p2 p3 p4
	\iupac{\chembeta}-Ala 0.281 0.296 1.798
	Gly 0.160 0.163 0.872
	Val 0.278 0.301 1.790
	Leu 0.293 0.299 1.681
	Ile 0.304 0.285 1.740
	Pro 0.296 0.290 1.186
	Phe 0.281 0.285 15.478
	Gln 0.236 0.254 1.169
	Lys 0.255 0.268 0.784
	Tyr 0.228 0.238 5.315

}\datatable

% get number of rows
\pgfplotstablegetrowsof{\datatable}
% subtract 1 because table indices start at 0
\pgfmathsetmacro{\Nrows}{\pgfplotsretval-1}
% for convenience, macro to store width of axis
\pgfmathsetlengthmacro{\MyAxisW}{10cm}

\begin{figure}[htbp]
	\centering
	\begin{adjustbox}{width=0.9\linewidth}
		\begin{tikzpicture}[
				cell/.style={ % style used for "table" cells
						draw,
						minimum width={\MyAxisW/(\Nrows+1)}, % +1 because -1 above
						minimum height=3ex,
						inner sep=0pt,
						outer sep=0pt,
						anchor=north west,
						font=\scriptsize
					}]
			\begin{axis}[
					name=ax,
					% so axis labels and ticklabels are not accounted for in size settings   
					scale only axis,
					width=\MyAxisW,
					height=4cm,
					% use a stacked bar char
					ybar,
					grid=both,
					grid style={line width=.2pt, draw=black!25},
					major grid style={line width=.3pt,draw=black!90},
					% set distance between yticks
					ytick={0,1,2},
					minor tick num=3,
					% only need left y-axis line
					axis y line=left,
					xtick=\empty, % we add the ticklabels as part of the table, so no xticks needed
					x axis line style={draw=none},
					% divide axis width by twice the number of rows, so that the whitespace between bars is the same as the bar width
					bar width={\MyAxisW/(6*\Nrows)},
					% and for that we need to make sure that the distance from the first/last tick to the axis edge is the same, so that there is a half a bar width of space
					enlarge x limits={abs={\MyAxisW/(2*\Nrows+2)}},
					ymin=0, ymax=2.1,
					ylabel={\footnotesize Indhold ift. prøve 1},
					xlabel={},
					tick label style={font=\footnotesize},
				]
				% because the x-values are not evenly spaced, used index as x-value instead
				\addplot +[black!69] table[x expr=\coordindex,y=p2] {\datatable};
				\label{datap2}

				\addplot +[black!29] table[x expr=\coordindex,y=p3] {\datatable};
				\label{datap3}

				\addplot +[black!52] table[x expr=\coordindex,y=p4] {\datatable};
				\label{datap4}
			\end{axis}

			%% Hardcoded %%
			% Show discontinuity in bars
			\coordinate (phe) at (6.75,4);
			\coordinate (tyr) at (9.75,4);
			\path[fill=black!52] ($(phe)+(-0.1,0.06)$) rectangle ($(phe)+(0.1,0.26)$) node[anchor=east,xshift=-3pt] {\scriptsize\num{15.478}};
			\draw ($(phe)+(-0.2,0)$) -- ($(phe)+(0.2,0)$);
			\draw ($(phe)+(-0.2,0.06)$) -- ($(phe)+(0.2,0.06)$);
			\path[fill=black!52] ($(tyr)+(-0.1,0.06)$) rectangle ($(tyr)+(0.1,0.26)$) node[anchor=east,xshift=-3pt] {\scriptsize\num{5.315}};
			\draw ($(tyr)+(-0.2,0)$) -- ($(tyr)+(0.2,0)$);
			\draw ($(tyr)+(-0.2,0.06)$) -- ($(tyr)+(0.2,0.06)$);
			%% %% %% %% %%

			% define a starter coordinate at the lower left corner of the axis
			\coordinate (c-0-0) at (ax.south west);

			% loop over the table
			\foreach [count=\j from 1] \i in {0,...,\Nrows}
				{
					% get element \i from the x-column, stored in \pgfplotsretval and add node with value
					\pgfplotstablegetelem{\i}{AA}\of\datatable
					\node [cell] (c-0-\j) at (c-0-\i.north east) {\pgfplotsretval};

					\pgfplotstablegetelem{\i}{p2}\of\datatable
					\node [cell] (c-1-\j) at (c-0-\j.south west) {\num{\pgfplotsretval}};

					\pgfplotstablegetelem{\i}{p3}\of\datatable
					\node [cell] (c-2-\j) at (c-1-\j.south west) {\num{\pgfplotsretval}};

					\pgfplotstablegetelem{\i}{p4}\of\datatable
					\node [cell] (c-3-\j) at (c-2-\j.south west) {\num{\pgfplotsretval}};
				}

			% add "legend" on the left
			\matrix [draw,nodes={cell,draw=none},anchor=north east,row sep=0pt,outer sep=0pt,inner ysep=0pt,inner xsep=-3pt] at (c-1-1.north west)
			{
				\node {\ref{datap2}}; & \node{P2}; \\
				\node {\ref{datap3}}; & \node{P3}; \\
				\node {\ref{datap4}}; & \node{P4}; \\
			};
		\end{tikzpicture}
	\end{adjustbox}
	\caption{Søjlediagram over relativt aminosyreindhold af prøve \numrange{2}{4} i forhold til prøve \num{1}, med tilhørende tabel hvori den eksakte værdi kan aflæses. Der bør noteres at to søjler overstier grafens akser. Lavet på baggrund af data i \cref{tab:area}.}
	\label{fig:relativeAA}
\end{figure}
