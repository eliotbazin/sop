\section*{Resumé}\addcontentsline{toc}{section}{Resumé}
I dette projekt undersøges, hvordan massespektrometri kan bruges til at afsløre snyd i honning, ved at analysere honningprøvers aminosyreindhold og -fordeling.
Et massespektrometer er en analysemetode der ved hjælp af ladede partiklers bevægen gennem elektriske eller magnetiske felter kan bestemme deres masse pr. ladning, og derved afgøre det analyserede stof.
Projektet vil gøre rede for honnings kemiske sammensætning, med fokus på aminosyrer, og for hvordan massespektrometre kan bruge disse ladede partiklers bevægelse til at afgøre stoffet.
Der vil i forbindelse med projektet blive lavet et forsøg, hvor aminosyreindholdet i nogle honningprøver måles ved brug af et massespektrometer,
og indholdet analyseret, for at kunne afgøre hvorvidt dette kan bruges til at afsløre snyd i honningen.
\newline Det konkluderes at massespektrometri, som bruges til at undersøge aminosyreindholdet i honning, er en god analysemetode til at afsløre snyd i honning,
da den er billigere end andre udbredte metoder som \isotope*{1,H}-NMR.
Men at det dog vil kræve bedre kortlægning af honnings aminosyreindhold afhængig af geografisk og botanisk oprindelse at kunne bestemme det nøjagtigt,
da honning er et naturprodukt, og der derfor vil være stor variation.
