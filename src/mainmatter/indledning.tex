\chapter*{\hypertarget{chap:indledning}{Indledning}}\addcontentsline{toc}{chapter}{Indledning}
\epigraphhead[12ex]{\epigraph{My son, eat thou honey, because \textit{it is} good}{\textit{Kong Salomon} - Ordsprogenes bog 24:13\nocite{bible}}}

Honning er det naturlige søde stof, der frembringes af \textit{Apis mellifera}-bier på grundlag af plantesaft, som bierne bearbejder og lader modne i honningtavler \parencite{honningbekDA}.
Der er et enormt globalt marked for honning, og det blev i 2022 vurderet til \num{9.01} milliarder USD og er antaget til at gro til \num{14.34} milliarder USD i 2031 \parencite{honeyMarket}.
\par Der sættes høje standarder til dette naturprodukt i særlig Danmark og EU, hvor der er hårde krav for produktionen og forarbejdningen af honning, og der må hverken tilføjes eller fjernes stoffer fra det naturligt producerede stof \parencite{honningbekDA,honningbekEU}.
Dog viste en rapport af EU kommisionen at \qty{46}{\percent} af \num{320} honningprøver importeret til EU i perioden 10/2021--02/2022, mistanke for snyd med honning \parencite{EUhoney}.
Forfalskningen indebærer oftest opblanding med billige sukkersirupper, men begrebet dækker over led i processen, der gør at honningen ikke er et \qty{100}{\percent} naturligt produkt, produceret af \textit{A. mellifera}-bier, udelukkende på naturlig plantesaft \parencite{adulterationhoney}.
Forfalskning kan også indebære forkert mærkning af honning, sådan at det sælges dyrere end hvad det er, og altså ikke har den korrekte geografiske og botaniske oprindelse \parencite{geohoney}.
\par Snyd er ikke bare et problem i honning importeret til EU, men Apimondia (\textit{the International Federation of Beekeepers’ Associations}) vil ved World Beekeeping Awards 2025, som afholdes i København, ikke længere uddele priser til honning \parencite{awardFraud}.
De har udtalt:
\par "\textit{We will celebrate honey in many ways at the Congress [\dots] but honey will no longer be a category, and thus no honey judging, in the World Beekeeping Awards. This change to remove honey as a category was necessitated by the inability to have honey fully tested for adulteration.}" \parencite{apimondia}.
\par Denne stigende mængde snyd har ført til forskning inden for metoder hvorpå forfalskningen kan afsløres.
Honningens kemiske sammensætning, og særligt aminosyreindhold, er interessant, da det kan sige rigtig meget om honningens geografiske såvel som botaniske oprindelse \parencite{geohoney}.
Til dette kan metoder som massespektrometri bruges, der kan fortælle noget om indholet af et stof i en opløsning.
\par Der vil altså i dette projekt blive undersøgt, hvordan massespektrometri kan anvendes til at afsløre forfalskning af honning, ved at måle indholdet af frie aminosyrer.
