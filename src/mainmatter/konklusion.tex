\chapter{Konklusion}\addcontentsline{toc}{chapter}{Konklusion}
Honning består primært af saccharider og kun en lille bestanddel andre stoffer som frie aminosyrer,
der fremkommer i plantesaft i forskellige mængde, afhængig af geografisk og botanisk oprindelse, og derfor siger meget om honningens autenticitet.
Et massespektrometer er et instrument der ved brug af ladede partiklers bevægelse gennem et magnetisk eller elektrisk felt, kan bestemme stoffet.
I klassiske massespektrometre blev stofferne opdelt efter impuls eller kinetisk energi i hhv. magnetiske og elektriske felter, mens et firpolet massespektrometer,
som gør brug af et oscillerende elektrisk felt, kan justerer spændingen for kun at lade partikler med det rette forhold mellem masse og ladning gennem.
\par Det var ved et forsøg, hvor aminosyreindholdet i honningprøver blev målt, muligt at bestemme de forfalskede honninger med opblandet sukkersirup.
Dog kan det være svært at bestemme små mængder snyd i honning ved den anvendte metode, da honning som naturprodukt har enormt stor variation i sit indhold.
Det vil anbefales at der i stedet for et firpolet massespektrometer andvendes et high resolution massespektrometer.
\par
Aminosyren prolin var den klart mest fremtrædende i honningen, og kan være en god indikator for autenticiteten af honning.
Det vil kræve dog kræve en bedre kortlægning af aminosyreindholdet af honning afhængig af deres geografiske og botaniske oprindelse at kunne afgøre hvorvidt honning er forfalsket eller ej.
\par Massespektrometri som bruges til at undersøge aminosyreindholdet, er en god analysemetode til at afsløre snyd i honning grundet evnen til præcist at angive mængden af disse.
Denne metode er billigere og kan give et mere præcist svar end andre metoder som \isotope*{1,H}-NMR.
