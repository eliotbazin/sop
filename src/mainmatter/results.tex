\section{Resultater}
\subsection{Identificering}
Af de fire chromatogrammer (se \cref{appsec:chrom}) valgtes interessante toppe som vil blive sammenlignet for de fire prøver.
De interessante toppe undersøgtes, og der blev ved undersøgelse af massespektre afgjort stoffet.
De toppe hvor der var høj sandsynlighed for at det var en binding med en aminosyre, kan ses i \cref{tab:id-substance}.
\begin{table}[htbp]
	\centering
	\caption{Tabel over indentificerede stoffer i de fire prøver ved GC-MS, indentificeret med \cite{nist23}. \emph{Match} beskriver, hvor godt det eksperimentelle massespektrum passer til det i databasen, og \emph{R. Match} beskriver hvor godt det i databasen passer til det eksperimentelle spektrum.}
	\begin{tabular}{S[table-format=2.3]lS[table-format=3.0]S[table-format=3.0]}\toprule
		RT     & Indentificeret stof                                          & Match & {R. Match} \\\midrule
		5.255  & \iupac{\chembeta-Alanine, N-ethocarbonyl-, ethyl ester}      & 884   & 886        \\
		5.448  & \iupac{Glycine, N-methyl-N-ethoxycarbonyl-, ethyl ester}     & 850   & 858        \\
		5.463  & \iupac{Glycine, N-(ethoxycarbonyl)-, ethyl ester}            & 724   & 827        \\
		5.635  & \iupac{\chembeta-Alanine, N-(ethoxycarbonyl)-, methyl ester} & 813   & 819        \\
		5.955  & \iupac{\L-Valine, N-ethoxycarbonyl-, ethyl ester}            & 892   & 894        \\
		6.001  & \iupac{\L-Alanine, N-ethoxycarbonyl-, ethyl ester}           & 910   & 928        \\
		6.382  & \iupac{\L-Leucine, N-ethoxycarbonyl-, pentyl ester}          & 836   & 857        \\
		6.478  & \iupac{\L-Isoleucine, N-ethoxycarbonyl-, ethyl ester}        & 884   & 887        \\
		6.803  & \iupac{\L-Proline, N-ethoxycarbonyl-, methyl ester}          & 920   & 921        \\
		7.077  & \iupac{\L-Proline, N-ethoxycarbonyl-, ethyl ester}           & 934   & 937        \\
		7.239  & \iupac{\L-Alanine, 3-cyano-, N-ethoxycarbonyl-, ethyl ester} & 812   & 852        \\
		7.468  & \iupac{\L-Proline, N-ethoxycarbonyl-}                        & 932   & 932        \\
		7.793  & \iupac{\L-Proline, N-ethoxycarbonyl-, ethyl ester}           & 696   & 791        \\
		8.681  & \iupac{\L-Phenylalnine, N-ethoxycarbonyl-, methyl ester}     & 888   & 900        \\
		8.909  & \iupac{\L-Phenylalanine, N-ethoxycarbonyl-, ethyl ester}     & 930   & 930        \\
		9.909  & \iupac{\L-Glutamine, N-ethoxycarbonyl-, methyl ester}        & 758   & 792        \\
		10.665 & \iupac{\L-Lysine, N,N'-di(ethoxycarbonyl)-, ethyl ester}     & 729   & 733        \\
		11.482 & \iupac{\L-Tyrosine, N,O-bis(ethoxycarbonyl)-, ethyl ester}   & 767   & 788        \\\bottomrule
	\end{tabular}
	\label{tab:id-substance}
\end{table}

\par I \cref{tab:id-substance} ses det at stoffet ved retentionstid \qty{7.077}{\minute} og \qty{7.793}{\minute} begge blev identificeret til \iupac{\L-Proline, N-ethoxycarbonyl-, ethyl ester}.
Det er ikke muligt at et stof har to forskellige retentionstider, så en af dem er nok blev fejlidentificeret.
Stoffet ved RT \qty{7.077}{\minute} har meget høj Match og R. Match faktor, hvorimod stoffet ved RT \qty{7.793}{\minute} har lavt match.
Dette tyder på stof nummer to nok ikke er \iupac{\L-Proline, N-ethoxycarbonyl-, ethyl ester}, men derimod et andet stof.
Dog har stoffet's massespektrum en stor top ved $m/z$ \num{70}, som er meget karakteristisk for prolin, og stoffet er derfor nok en anden prolinforbindelse.
\subsection{Kvantificering}
\par For hver stof i \cref{tab:id-substance} valgtes der én target ion og \numrange{1}{2} qualifier ions (se \cref{tab:area}), som vil bruges til at bestemme den relative mængde af stofferne i prøven.
Der blev kigget på ionchromatogrammer for at vælge eksklusive ioner, der vil minimere støj i målingerne.
Toppene i ionchromatogrammerne blev integreret for at bestemme den målte mængde af stofferne.
Derefter blev der delt med den numeriske værdi af massen af den afmålte mængde honning, fra \cref{tab:samples}, for at mindske afvigelser.
De korrigerede integrerede arealer for stofferne kan ses i \cref{tab:area}.
\begin{table}[htbp]
	\centering
	\caption{Tabel over areal for de identificerede stoffer i de fire prøver, samt \emph{target ion} (TI) og \emph{qualifier ioner} (QI) brugt til at bestemme dette. Arealerne er fundet ved brug af \citetitle{MHquant} \parencite{MHquant}.}
	\begin{tabular}{S[table-format=2.3]S[table-format=3.0]S[table-format=3.0]S[table-format=3.0]S[table-format=7.0]S[table-format=7.0]S[table-format=7.0]S[table-format=7.0]}\toprule
		{\multirow{2}{*}{RT}} & {\multirow{2}{*}{TI}} & {\multirow{2}{*}{QI1}} & {\multirow{2}{*}{QI2}} & \multicolumn{4}{c}{Areal}                                                                                  \\\cmidrule{5-8}
		                      &                       &                        &                        & \multicolumn{1}{c}{P. 1}  & \multicolumn{1}{c}{P. 2} & \multicolumn{1}{c}{P. 3} & \multicolumn{1}{c}{P. 4} \\\midrule
		5.255                 & 116                   & 44                     & 29                     & 46747                     & 13726                    & 14691                    & 80729                    \\
		5.448                 & 116                   & 44                     & 88                     & 27734                     & 7160                     & 7245                     & 28819                    \\
		5.463                 & 102                   & 175                    & {--}                   & 16887                     & 4431                     & 4473                     & 24734                    \\
		5.635                 & 115                   & 98                     & 102                    & 8003                      & 2688                     & 2810                     & 14170                    \\
		5.955                 & 144                   & 129                    & 72                     & 27205                     & 7552                     & 8201                     & 48695                    \\
		6.001                 & 115                   & 102                    & 98                     & 6943                      & 1313                     & 897                      & 8822                     \\
		6.382                 & 158                   & 102                    & {--}                   & 15942                     & 4676                     & 4767                     & 26798                    \\
		6.478                 & 158                   & 102                    & 74                     & 14119                     & 4293                     & 4027                     & 24560                    \\
		6.803                 & 142                   & 70                     & 114                    & 134734                    & 73922                    & 84971                    & 181752                   \\
		7.077                 & 142                   & 70                     & 114                    & 1802468                   & 530274                   & 531468                   & 2081355                  \\
		7.239                 & 141                   & 69                     & {--}                   & 9243                      & 2187                     & 2605                     & 23831                    \\
		7.468                 & 142                   & 114                    & 70                     & 271403                    & 50524                    & 24012                    & 342287                   \\
		7.793                 & 188                   & 142                    & 98                     & 13168                     & 3960                     & 3247                     & 30481                    \\
		8.681                 & 162                   & 131                    & 91                     & 716                       & 370                      & 420                      & 13722                    \\
		8.909                 & 176                   & 102                    & 91                     & 27330                     & 7501                     & 7568                     & 420367                   \\
		9.909                 & 173                   & 84                     & 128                    & 8774                      & 2074                     & 2224                     & 10258                    \\
		10.665                & 156                   & 128                    & 84                     & 64702                     & 16476                    & 17326                    & 50697                    \\
		11.482                & 192                   & 107                    & 135                    & 9200                      & 2100                     & 2193                     & 48900                    \\\bottomrule
	\end{tabular}
	\label{tab:area}
\end{table}

\par For at finde prøvernes aminosyreindhold i forhold til prøve \num{1}, blev arealerne under stoffer med samme aminosyre summeret
og derefter blev indholdet af de enkelte aminosyrer delt med indholdet i prøve \num{1}, som var kontrolprøven af \qty{100}{\percent} dansk honning (se \cref{tab:samples}).
Det gøres da vi antager at kontrolhonningen er vores standard, og det er afvigelser fra denne vi er interesserede i.
Resultatet af dette kan ses i \cref{fig:relativeAA}.
% read in data
\pgfplotstableread{
	AA p2 p3 p4
	\iupac{\chembeta}-Ala 0.281 0.296 1.798
	Gly 0.160 0.163 0.872
	Val 0.278 0.301 1.790
	Leu 0.293 0.299 1.681
	Ile 0.304 0.285 1.740
	Pro 0.296 0.290 1.186
	Phe 0.281 0.285 15.478
	Gln 0.236 0.254 1.169
	Lys 0.255 0.268 0.784
	Tyr 0.228 0.238 5.315

}\datatable

% get number of rows
\pgfplotstablegetrowsof{\datatable}
% subtract 1 because table indices start at 0
\pgfmathsetmacro{\Nrows}{\pgfplotsretval-1}
% for convenience, macro to store width of axis
\pgfmathsetlengthmacro{\MyAxisW}{10cm}

\begin{figure}[htbp]
	\centering
	\begin{adjustbox}{width=0.9\linewidth}
		\begin{tikzpicture}[
				cell/.style={ % style used for "table" cells
						draw,
						minimum width={\MyAxisW/(\Nrows+1)}, % +1 because -1 above
						minimum height=3ex,
						inner sep=0pt,
						outer sep=0pt,
						anchor=north west,
						font=\scriptsize
					}]
			\begin{axis}[
					name=ax,
					% so axis labels and ticklabels are not accounted for in size settings   
					scale only axis,
					width=\MyAxisW,
					height=4cm,
					% use a stacked bar char
					ybar,
					grid=both,
					grid style={line width=.2pt, draw=black!25},
					major grid style={line width=.3pt,draw=black!90},
					% set distance between yticks
					ytick={0,1,2},
					minor tick num=3,
					% only need left y-axis line
					axis y line=left,
					xtick=\empty, % we add the ticklabels as part of the table, so no xticks needed
					x axis line style={draw=none},
					% divide axis width by twice the number of rows, so that the whitespace between bars is the same as the bar width
					bar width={\MyAxisW/(6*\Nrows)},
					% and for that we need to make sure that the distance from the first/last tick to the axis edge is the same, so that there is a half a bar width of space
					enlarge x limits={abs={\MyAxisW/(2*\Nrows+2)}},
					ymin=0, ymax=2.1,
					ylabel={\footnotesize Indhold ift. prøve 1},
					xlabel={},
					tick label style={font=\footnotesize},
				]
				% because the x-values are not evenly spaced, used index as x-value instead
				\addplot +[black!69] table[x expr=\coordindex,y=p2] {\datatable};
				\label{datap2}

				\addplot +[black!29] table[x expr=\coordindex,y=p3] {\datatable};
				\label{datap3}

				\addplot +[black!52] table[x expr=\coordindex,y=p4] {\datatable};
				\label{datap4}
			\end{axis}

			%% Hardcoded %%
			% Show discontinuity in bars
			\coordinate (phe) at (6.75,4);
			\coordinate (tyr) at (9.75,4);
			\path[fill=black!52] ($(phe)+(-0.1,0.06)$) rectangle ($(phe)+(0.1,0.26)$) node[anchor=east,xshift=-3pt] {\scriptsize\num{15.478}};
			\draw ($(phe)+(-0.2,0)$) -- ($(phe)+(0.2,0)$);
			\draw ($(phe)+(-0.2,0.06)$) -- ($(phe)+(0.2,0.06)$);
			\path[fill=black!52] ($(tyr)+(-0.1,0.06)$) rectangle ($(tyr)+(0.1,0.26)$) node[anchor=east,xshift=-3pt] {\scriptsize\num{5.315}};
			\draw ($(tyr)+(-0.2,0)$) -- ($(tyr)+(0.2,0)$);
			\draw ($(tyr)+(-0.2,0.06)$) -- ($(tyr)+(0.2,0.06)$);
			%% %% %% %% %%

			% define a starter coordinate at the lower left corner of the axis
			\coordinate (c-0-0) at (ax.south west);

			% loop over the table
			\foreach [count=\j from 1] \i in {0,...,\Nrows}
				{
					% get element \i from the x-column, stored in \pgfplotsretval and add node with value
					\pgfplotstablegetelem{\i}{AA}\of\datatable
					\node [cell] (c-0-\j) at (c-0-\i.north east) {\pgfplotsretval};

					\pgfplotstablegetelem{\i}{p2}\of\datatable
					\node [cell] (c-1-\j) at (c-0-\j.south west) {\num{\pgfplotsretval}};

					\pgfplotstablegetelem{\i}{p3}\of\datatable
					\node [cell] (c-2-\j) at (c-1-\j.south west) {\num{\pgfplotsretval}};

					\pgfplotstablegetelem{\i}{p4}\of\datatable
					\node [cell] (c-3-\j) at (c-2-\j.south west) {\num{\pgfplotsretval}};
				}

			% add "legend" on the left
			\matrix [draw,nodes={cell,draw=none},anchor=north east,row sep=0pt,outer sep=0pt,inner ysep=0pt,inner xsep=-3pt] at (c-1-1.north west)
			{
				\node {\ref{datap2}}; & \node{P2}; \\
				\node {\ref{datap3}}; & \node{P3}; \\
				\node {\ref{datap4}}; & \node{P4}; \\
			};
		\end{tikzpicture}
	\end{adjustbox}
	\caption{Søjlediagram over relativt aminosyreindhold af prøve \numrange{2}{4} i forhold til prøve \num{1}, med tilhørende tabel hvori den eksakte værdi kan aflæses. Der bør noteres at to søjler overstier grafens akser. Lavet på baggrund af data i \cref{tab:area}.}
	\label{fig:relativeAA}
\end{figure}

\par Som det kan ses på \cref{fig:relativeAA} har prøve \num{4} højere indhold af de fleste aminosyrer det var muligt at detektere, på nær glycin og lysin, sammenlignet med prøve 1.
Derudover kan det også ses på figuren at prøve 2 og 3 er omkring \qty{25}{\percent} af kontrolprøven, hvilket passer nogenlunde overens med blandingsforholdende i \cref{tab:samples}.
\begin{wrapfigure}{O}{0.56\linewidth}
	\centering
	\begin{tikzpicture}[scale=0.6]
		\foreach \y [count=\n] in {
				{0.02831,0.02725,0.02922,0.03701},
				{0.01781,0.00974,0.01013,0.01129},
				{0.01086,0.01033,0.01141,0.01413},
				{0.00636,0.00640,0.00663,0.00778},
				{0.00564,0.00588,0.00560,0.00713},
				{0.88682,0.90137,0.89563,0.76484},
				{0.01119,0.01077,0.01111,0.12596},
				{0.00350,0.00284,0.00309,0.00298},
				{0.02583,0.02255,0.02411,0.01471},
				{0.00367,0.00287,0.00305,0.01419},
			} {
				% column labels
				\foreach \a [count=\i] in {P1,P2,P3,P4} {
						\node at ($2.5*(\i,0)-(0.75,0)$) {\a};
					}
				% heatmap tiles
				\foreach \x [count=\m] in \y {
					\pgfmathparse{100*\x}
					\node[fill=yellow!\pgfmathresult!purple!80!white, minimum size=6mm, text=white] (c-\n-\m) at ($2.5*(\m,0)+(-.65,-\n)$) {\small\num{\x}};
				}
			}

		% row labels
		\foreach \a [count=\i] in {\iupac{\chembeta-Ala},Gly,Val,Leu,Ile,Pro,Phe,Gln,Lys,Tyr} {
				\node at (-1em,-\i) {\parbox{3em}{\a}};
			}

		% Border
		\draw (c-1-1.north west)rectangle(c-10-4.south east);

		% Legend
		\pgfdeclareverticalshading{bar}{100bp}{
			color(0bp)=(yellow!0!purple!80!white);
			color(25bp)=(yellow!0!purple!80!white);
			color(30bp)=(yellow!10!purple!80!white);
			color(35bp)=(yellow!20!purple!80!white);
			color(40bp)=(yellow!30!purple!80!white);
			color(45bp)=(yellow!40!purple!80!white);
			color(50bp)=(yellow!50!purple!80!white);
			color(55bp)=(yellow!60!purple!80!white);
			color(60bp)=(yellow!70!purple!80!white);
			color(65bp)=(yellow!80!purple!80!white);
			color(70bp)=(yellow!90!purple!80!white);
			color(75bp)=(yellow!100!purple!80!white);
			color(100bp)=(yellow!100!purple!80!white)
		}

		\shade[shading=bar]($(c-10-4.south east)+(.5,0)$)rectangle($(c-1-4.north east)+(1,-0.03)$);

		\draw ($(c-10-4.south east)+(1,0)$) -- ($(c-10-4.south east)+(1,10)$);
		\draw ($(c-10-4.south east)+(1.1,0)$) node[anchor=west] {$0.00$} -- ($(c-10-4.south east)+(0.5,0)$) -- ($(c-10-4.south east)+(0.5,10)$) -- +(0.6,0) node[anchor=west] {$1.00$};
		\draw ($(c-10-4.south east)+(0.9,2.5)$) -- +(0.2,0) node[anchor=west] {$0.25$};
		\draw ($(c-10-4.south east)+(0.9,5)$) -- +(0.2,0) node[anchor=west] {$0.50$};
		\draw ($(c-10-4.south east)+(0.9,7.5)$) -- +(0.2,0) node[anchor=west] {$0.75$};

	\end{tikzpicture}
	\caption{Heatmap over aminosyrefordeling i de fire prøver som andel af samles aminosyreindhold i prøven. Lavet på baggrund af data i \cref{tab:area}.}
	\label{fig:aa-comp}
\end{wrapfigure}

\par For at finde aminosyrefordelingen i de fire prøve blev aminosyreindholdet i hver af de fire prøver summeneret, og hver aminosyres andel udregnet.
Resultatetet af dette kan ses i \cref{fig:aa-comp}. Det ses her at prolin er den klart dominerende aminosyre i honning, hvilket stemmer overens med blandt andet \textcite{AAhoney,AAhoneyPl}.
Derudover ses det også, at det relative prolinindhold i prøve 4, honningen fra ikke-EU lande, er lavere end i de andre prøver, men at det istedet indeholder højere relativt indhold phenylalanin.
Dette ses også på \cref{fig:relativeAA}, hvor prøve 4 indeholder over \num{15} gange så meget phenylalanin i forhold til prøve 1.
I \cref{fig:relativeAA} fremgår det også, at prøve 4 indeholder over \num{5} gange så meget tyrosin som prøve 1, men på \cref{fig:aa-comp} ses det at tysosinindholdet generelt er meget lavt i prøverne.
\par På \cref{fig:aa-comp} ses det at prøve \numrange{1}{3} har omtrent samme aminosyrefordeling, hvilket også var det der forventes, da prøve 2 og 3 er opblandinger af prøve 1.
Dog afviger opblandingernes forhold af glycinindhold.
\subsection{Forsøgsopsamling}
Generelt er det ud fra \cref{fig:relativeAA,fig:aa-comp} muligt at bestemme at prøve 2 og 3 er opblandinger af prøve 1.
Det kan umiddelbart være svært at sammenligne prøve 1 og 4 (dansk og ikke-EU honning), da aminosyreindholdet er forskelligt.
Dog kan nogle aminosyre som glycin, glutamin, prolin og lysin bruges som indikatorer.
De er alle forholdvist tæt i absolut indhold, og det relative indhold af særligt glutamin ligger nært.
\par Særligt prolin kan fungere godt som indikator da dette er den klart mest indeholdte aminosyre i honning.
Den har dog det problem at dens koncentration falder under oplagring, og er derfor brugt som indikator for honningens friskhed \parencite{VonDerOhe}.
\par I forsøget blev bestemt et prolinindhold på \qtyrange{76}{90}{\percent} af de totale aminosyre, hvor studierne \textcite{geohoney,aaPureHoney} har bestemt et prolinindhold på \qtyrange{50}{85}{\percent} af de totale aminosyre.
I de andre studier lykkedes det dog at finde \num{26} aminosyrer, hvorimod dette forsøg kun fandt \num{10}. Det højere antal af fundne aminosyrer gør også det totale aminosyreindhold højere, hvilket vil gøre prolins andel mindre.
Altså passer det eksperimentelt bestemte prolinindhold fint med andet litteratur.
\par Prøve 3 har vist et generelt højere indhold af aminosyrer end prøve 2, selvom det kunne forventes der var aminosyrer i agavesiruppen \parencite{agaveAA} og ikke i glucosesirup, da disse ikke burde fremkomme i produktionen \parencite{glucoseManufacture}.
Dette kunne skyldes upræcis opmåling ved opblanding af prøverne ud fra \cref{tab:samples}.
Studiet \textcite{prolineHoney}, har vist hvordan prolinindholdet falder ved tilsætning af sukkersirupper til honning.
Studiet fandt at prolinindholdet falder i højere grad ved tilsætning af fructose end ved tilsætning af glucose, og da agavesirup består af over \qty{50}{\percent} fructose \parencite{agaveSyrup} kunne dette også have været være en medvirkende faktor.
\par I prøve 4 var indholdet af phenylalanin \num{15} gange højere end i den danske honning (se \cref{fig:relativeAA}).
Dette kan være betydeligt for personer der lider af \emph{fenylketonuri} (PKU), som er en genetisk lidelse, der forårsager problemer med nedbrydelsen af phenylalanin.
Disse mennesker kan ikke nedbryde phenylalanin og den vil derfor ophobe sig i kroppen, hvilket kan være skadeligt for hjernen og nervesystemet. \parencite{PKU}
\par Disse konklusioner kræver, at stofferne blev korrekt identificeret, men i \cref{tab:id-substance} ses det at ikke alle stofferne havde fantastisk match og reverse match med databasen \textcite{nist23}.
Altså kunne der godt være forekommet misidentifikation af nogle af stofferne.
