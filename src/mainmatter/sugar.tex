\section{Carbohydrater}\label{sec:sugar}
Ordet sukker bliver i daglig tale primært brugt som navnet på carbohydratet sucrose, men betegnelses gælder typisk for mono- eller disaccharider \parencite{sugardeff}.
Carbohydrater består af carbon, hydrogen og oxygen, og indeholder \ch{H} og \ch{O} i samme forhold som \ch{H2O}, altså $2\!:\!1$ \parencite{basisB}.
Længere carbohydrater som stivelse og cellulose er polymere af monosaccharider bundet ved glykosidbindinger, som er en kovalent binding dannet ved en kondensationsreaktion \parencite{sugarbond}.
\subsection{Monosaccharider}
De primære saccharider i honning er monosacchariderne fructose og glucose, som bliver produceret ved hydrolyse af disaccharider.
Fructose er den primære årsag til honningens sødhed \parencite{fuctosehealth}, og den er i næsten alt honninger den primære saccharid \parencite{sugarhoney}.
Forholdet mellem fructose og glucose afhænger meget af kilden til nektar \parencite{geohoney}, og denne ratio kan bruges til at sige hvor hurtigt honning størkner \parencite{honeycrystal}.
\par Monosaccharider navngives efter deres funktionelle grupper (aldehyd (\ch{-CHO}) eller keton (\ch{-CO{-}})), og antallet af carbonatomer, som varierer fra \numrange{3}{8}.
Mange simple sukre findes kun meget lidt naturligt da de hurtigt binder sig til polysaccharider.
Undtagelser til dette er glucose og fructose der fremkommer naturligt i store mængder \parencite{physwoodplant}.
Alle monosacchariderne, med undtagelse af dihydroxyacetone, har mindst et chiralt carbonatom, og der findes altså to eller flere stereoisomere af dem.
Ved navngivning af monosaccharider kigges kun på det asymetriske carbonatom længst fra carbonet i carbonylgruppen.
Den betegnes med \iupac{\L} eller \iupac{\D}, men af de naturligt fremkomne sukre findes de næsten alle kun på \iupac{\D}-form \parencite{physwoodplant}.
\subsection{Di- og polysaccharider}
Indholdet af disaccharider i honning er omkring \qty{7}{\percent}, og er altså ikke nær så højt som indholdet af monosaccharider (se \cref{fig:honey-comp}).
Indholdet af sucrose i naturlig honning er kun lidt over \qty{1}{\percent}, og nogle mener denne faktor er kan være vigtig at kigge på for at afsløre snyd \parencite{saccharosehoney}, mens andre mener den ikke kan bruges \parencite{biochemprophoney}.
Polysaccharider er kæder af mere end to monosaccharider, og der findes kun meget få dem i honning. Dog har honning lavet af honningdug en højere koncentration af korte polysaccharider \parencite{sugarhoney}.
