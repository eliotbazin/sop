\chapter{Massespektrometri}
\emph{Massespektrometri} (MS) er en kemisk analysemetode som går ud på at ionisere et molekyle og dets fragmenter, adskille dem ud fra forholdet mellem masse og ladning ($m/z$) og kvantitativt bestemme ionintensiteten ved en given $m/z$.
Alle komponenter i et massespektrometre opererer under højt vakuum for at undgå forstyrrelser af ionerne (se \cref{subsec:freelength}).
Massespektrometri er, i modsætning til andre kemiske analysemetoder, såsom \emph{kernemagnetisk resonans} (NMR) eller infrarød spektroskopi (IR), destruktiv da den forbruger analytten.
Dog kræver analysemetoden meget meget lidt (få mikrogram) af analytten for at få et godt resultat \parencite{mstextbook}.
\par I massespektrometri bruges forholdet $m/z$ til at beskrive resultaterne.
Den kan tænkes som atomar masse over elementarladninger for et molekyle \parencite{chemana}.
Det er en opfundet størrelse brugt i stedet for en SI-enhed som \unit[per-mode=power]{\kilogram\per\coulomb} \parencite{mstextbook}.
\subsection{Massespektrum}
\begin{wrapfigure}{O}{0.5\linewidth}
	\centering
	\includesvg[width=.95\linewidth,pretex=\tiny]{graphics/pictures/massspektrum.svg}
	\caption{Eksempel på eksperimentielt fremstillet massespektrum. Fremstillet i \cite{nist23}.}
	\label{fig:examplems}
\end{wrapfigure}

Et massespektrometer vil producere et massespektrum som viser ionintensiteten ved en given $m/z$ værdi (se \cref{fig:examplems}).
Hver top vil være et fragment af analytten (se \cref{sec:electronion}).
De mindre toppe omkring de større skyldes isotopfordelingen af atomerne.
Historisk blev aflæsningen af massespektre gjort med opslagsværker som \citetitle{massspectra} \parencite{massspectra},
men i nutiden bruges elektroniske søgebiblioteker som \citetitle{nist23} \parencite{nist23}.
\subsection{Den fri middelvejslængde}\label{subsec:freelength}
Den fri middelvejlængde er et mål for, hvor langt et molekyle i en gas kan bevæge sig uden at støde ind i et andet molekyle \parencite{knudsenstrømning}.
I et massespektrometer vil man gerne have denne så stor som muligt, da ionerne ved kollision vil ændre deres bane og eventuelt lave uønskede reaktioner der ville give støj i målingen \parencite{massspec}.
Denne størrelse er givet ved:
\begin{equation}
	L = \frac{kT}{\sqrt{2}p\sigma}
\end{equation}
hvor $k$ er Boltzmannkonstanten, $T$ er den absolute temperatur, $p$ er trykket (i \unit{\pascal}) og $\sigma$ er kollisionstvæsnittet givet ved $\sigma = \pi d^2$,
hvor $d$ er summen af det stillestående molekyles og ionens radius \parencite{massspec}.
Massespektrometre arbejder derfor ved meget lavt tryk (mindst \qty{e-4}{\pascal}), da dette minimerer de uønskede effekter \parencite{mstextbook}.

