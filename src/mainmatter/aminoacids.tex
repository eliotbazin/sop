\section{Aminosyrer}
\begin{wrapfigure}{O}{0.3\linewidth}
	\centering
	\chemfig{[2]R-C_\chemalpha(-[0]H)(-[4]NH_2)-COOH}
	\caption{Fischer-projektion af den generelle struktur af \iupac{\L-\chemalpha-aminosyrer} på deres uioniserede form.}
	\label{fig:std-amino}
\end{wrapfigure}

En aminosyre er et molekyle der som minimum består af en carboxylgruppe (\ch{-COOH}) og en amingruppe (\ch{-NH2})\footnotemark{}.
\footnotetext{Her er vist en primær amingruppe som også er den mest normale i aminosyrer, men standartaminosyren prolin har en sekundær amingruppe (\raisebox{0.3\baselineskip}{\tiny\chembelow{-N -}{H}}).}
\iupac{\chemalpha-aminosyrer} er aminosyrer hvorpå amingruppen og carboxylgruppen er bundet til det samme $sp^3$-hybridiserede carbonatom (\ch{C_\chemalpha}).
Herpå er der også bundet et hydrogenatom (\ch{H}) og en organisk sidegruppe (\ch{-R}) som er den der adskilder aminosyrerne \parencite{aminoBrit}.
\iupac{\chemalpha-carbonatomet} er altså et chiralt center,\footnotemark{} hvilket betyder der findes enantiomere af aminosyrerne, men af de 20 forskellige aminosyrer, som indgår i opbygningen af proteiner, fremkommer de alle kun som \iupac{\L-enantiomere} \parencite{basisA}.
\footnotetext{På nær for aminosyren glycin som er achiral, da sidegruppen $\ch{R}=\ch{H}$.}
Disse standartaminosyrer kaldes \iupac{\L-\chemalpha-aminosyrer} og har den generelle struktur som ses på \cref{fig:std-amino}.
De 20 standartaminosyrers navne og struktur kan ses i \cref{tab:stdaminacid}.
% Set settings for table
\setchemfig{atom style={scale=0.7},atom sep=0.65cm,cram width=0.6ex,cram dash sep=1.4pt, cram dash width=0.7pt}
\definesubmol{a}{O=[:-90](-[:-30]OH)}
\begin{table}[hbt]
	\centering
	\caption{Tabel over de 20 standard aminosyrer, deres nominelle masse og deres struktur på deres uioniserede form. Stereodeskriptoren '\iupac{\L}' er udeladt i navnene. \parencite{aminoname,nist23}}
	\setlength{\extrarowheight}{4pt}
	\begin{tabular}{>{\small}wl{1.9cm} >{\small}wr{1.1cm} wc{3.4cm} | >{\small}wl{1.9cm} >{\small}wr{1.1cm} wc{3.4cm}}\toprule
		{\normalsize Navn} & {\normalsize $m$} & Struktur                                                                                                 &
		{\normalsize Navn} & {\normalsize $m$} & Struktur                                                                                                   \\\midrule
		Alanin             & 89                & \chemfig{!a-[:210](-[:150]H_3C)<[:-90]NH_2}                                                              &
		Cystein            & 121               & \chemfig{!a-[:210](<[:-90]NH_2)-[:150]-[:210]HS}                                                           \\
		Glycin             & 75                & \chemfig{!a-[:210]-[:150]NH_2}                                                                           &
		Glutamin           & 147               & \chemfig{!a-[:210](<[:-90]NH_2)-[:150]-[:210]-[:150](=[:90]O)-[:210]H_2N}                                  \\
		Isoleucin          & 131               & \chemfig{!a-[:210](<[:-90]NH_2)-[:150](<:[:90]CH_3)-[:210]-[:150]H_3C}                                   &
		Serin              & 105               & \chemfig{!a-[:210](<[:-90]NH_2)-[:150]-[:210]HO}                                                           \\
		Leucin             & 131               & \chemfig{!a-[:210](<[:-90]NH_2)-[:150]-[:210](-[:-90]CH_3)-[:150]H_3C}                                   &
		Threonin           & 119               & \chemfig{!a-[:210](<[:-90]NH_2)-[:150](<[:90]OH)-[:210]H_3C}                                               \\
		Methionin          & 149               & \chemfig{!a-[:210](<[:-90]NH_2)-[:150]-[:210]-[:150]S-[:210]H_3C}                                        &
		Tyrosin            & 181               & \chemfig{!a-[:210](<[:-90]NH_2)-[:150]-[:210]*6(=-=(-OH)-=-)}                                              \\
		Prolin             & 115               & \chemfig{!a-[:210]*5(----NH-[,,1])}                                                                      &
		Asparaginsyre      & 133               & \chemfig{!a-[:210](<[:-90]NH_2)-[:150]-[:210](=[:-90]O)-[:150]OH}                                          \\
		Valin              & 117               & \chemfig{!a-[:210](<[:-90]NH_2)-[:150](-[:90]CH_3)-[:210]H_3C}                                           &
		Glutaminsyre       & 147               & \chemfig{!a-[:210](<[:-90]NH_2)-[:150]-[:210]-[:150](=[:90]O)-[:210]OH}                                    \\
		Phenylalanin       & 165               & \chemfig{!a-[:210](<[:-90]NH_2)-[:150]-[:210]*6(=-=-=-)}                                                 &
		Arginin            & 174               & \chemfig{!a-[:210](<[:-90]NH_2)-[:150]-[:210]-[:150]-[:210]\chembelow{N}{H}-[:150](=[:90]NH)-[:210]H_2N}   \\
		Tryptophan         & 204               & \chemfig{!a-[:210](<[:-90]NH_2)-[:150]-[:210]*5(-*6(-=-=--)=-\chembelow{N}{H}-=)}                        &
		Histidin           & 155               & \chemfig{!a-[:210](<[:-90]NH_2)-[:150]-[:210]*5(-N=-\chembelow{N}{H}-=)}                                   \\
		Asparagin          & 133               & \chemfig{!a-[:210](<[:-90]NH_2)-[:150]-[:210](=[:-90]O)-[:150]H_2N}                                      &
		Lysin              & 146               & \chemfig{!a-[:210](<[:-90]NH_2)-[:150]-[:210]-[:150]-[:210]-[:150]H_2N}                                    \\\bottomrule
	\end{tabular}
	\label{tab:stdaminacid}
\end{table}
% Change settings back to default
\setchemfig{atom style={scale=1},atom sep=3em,cram width=1.5ex,cram dash sep=2pt, cram dash width=1pt}

\subsection{Syre-base egenskaber}
De frie aminosyrer vil ved en neutral pH primært være på deres dipolære form (som zwitterioner) og mindre på deres uinoniserede form.
I de dipolære aminosyrer vil amingruppen være protoniseret (\ch{-NH3+}) og carboxylgruppen være deprotoniseret (\ch{-COO-}).
Aminosyrernes ioniseringstrin vil ændres afhængig af pH i opløsningen \parencite{aminoBrit,chemana}.
