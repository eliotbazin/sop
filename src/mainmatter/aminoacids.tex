\section{Aminosyrer}
En aminosyre er et molekyle der som minimum består af en carboxylgruppe (\ch{-COOH}) og en amingruppe (\ch{-NH2})\footnotemark{}.
\footnotetext{Her er vist en primær amingruppe som også er den mest normale i aminosyrer, men standartaminosyren prolin har en sekundær amingruppe (\raisebox{0.3\baselineskip}{\tiny\chembelow{-N -}{H}}).}
\iupac{\chemalpha-aminosyrer} er aminosyrer hvorpå amingruppen og carboxylgruppen er bundet til det samme $sp^3$-hybridiserede carbonatom (\ch{C_\chemalpha}).
Herpå er der også bundet et hydrogenatom (\ch{H}) og en organisk sidegruppe (\ch{-R}) som er den der adskilder aminosyrerne \parencite{aminoBrit}.
\iupac{\chemalpha-carbonatomet} er altså et chiralt center,\footnotemark{} hvilket betyder der findes enantiomere af aminosyrerne, men af de 20 forskellige aminosyrer, som indgår i opbygningen af proteiner, fremkommer de alle kun som \iupac{\L-enantiomere} \parencite{basisA}.
\footnotetext{På nær for aminosyren glycin som er achiral, da sidegruppen $\ch{R}=\ch{H}$.}
Disse standartaminosyrer kaldes \iupac{\L-\chemalpha-aminosyrer} og har den generelle struktur som ses på \cref{fig:std-amino}.
De 20 standartaminosyrers navne og struktur kan ses i \cref{tab:stdaminacid}.
