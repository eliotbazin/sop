\section{Klassisk massespektrometer}
Massespektrometre der gjorde brug af magnet eller elektrisk felter var de første tilgængelige instrumenter \parencite{mstextbook}.
De fungerer ved at afbøje ionerne i et enten magnet eller elektrisk felt og ud fra radiusen ionerne bevæger sig ved bestemme deres $m/z$.
Begge versioner er meget følsom over for ionernes hastighed, og vil derfor efter ionkilden have et hastighedsfilter \parencite{orbitAstx}.
\subsection{Magnetfelt}
Ladede partikler, som bevæger sig i et \emph{magnetfelt} ($B$-felt) vil blive påvirket af en kraft kaldet Lorentzkraften \parencite{orbitAstx}.
For en ladet partikel med ladningen $q$, som bevæger sig med farten $v$ vinkelret på et konstant magnetfelt $B$ kan kraftpåvirkningen beskrives ved:
\begin{equation}
	F_B = qvB.
\end{equation}
Ionen vil følge en cirkulær bane med radius $r$, hvor kraften af magnetfeltet $F_B$ er lig centripetalkraften $F_c$ \parencite{orbitAstx}. Altså får vi:
\begin{equation}\label{eq:magfield}
	qvB = \frac{mv^2}{r}\quad\text{eller}\quad mv = qBr.
\end{equation}
Det ses altså at det magnetiske felt vil adskille partikler med samme ladning, afhængig af deres impuls.
Altså adskiller MS med $B$-felt ikke direkte ionerne på baggrund af deres masse, men derimod på baggrund af deres impuls,
men hvis de accelereres til en kendt hastighed, er effekten adskillelse af ionerne på baggrund af massen.
\par Gyroradiusen er radiusen af den bane en ladet partikel vil bevæge sig i, i et $B$-felt. Ved \cref{eq:magfield} fås:
\begin{equation}
	r = \frac{mv}{qB}.
\end{equation}
\subsection{Elektrisk felt}\label{subsec:e-felt}
Alle ladede partikler i et \emph{elektrisk felt} ($E$-felt) vil blive påvirket af kraft givet ved:
\begin{equation}
	F_E = qE.
\end{equation}
hvor $q$ er partiklens ladning og $E$ den elektriske feltstyrke \parencite{orbitAstx}.
Ionen vil også her følge en cirkulær bane med radius $r$, hvor kraften af det elektriske felt $F_E$ er lig centripetalkraften $F_c$ \parencite{mstextbook}. Altså får vi:
\begin{equation}\label{eq:elfield}
	qE = \frac{mv^2}{r}\quad\text{eller}\quad mv^2 = qEr.
\end{equation}
Her ses det at $E$-feltet vil adskille partikler med samme ladning, afhængig af deres kinetiske energi \parencite{massspec}.
Dette er den største fundementale forskel på klassiske massespektrometre med $B$-felt og med $E$-felt.
\par Her kan bevægelsesradiusen for partiklen i et $E$-felt findes ud fra \cref{eq:elfield} og der fås:
\begin{equation}
	r = \frac{mv^2}{qE}.
\end{equation}
