\section{Elektron ionisering}\label{sec:electronion}
Her vil blive gennemgået \emph{elektron ionisering} (EI) som metode til at ionisere molekylet, da det er denne metode der gør sig gældende for det benyttede massespektrometer. % @TODO reference til forsøg
Andre metoder til ionisering er \emph{kemisk ionisering} (CI), \emph{Fast Atom Bombardment} (FAB), og \emph{Electro Spray Ionisering} (ESI)\space\parencite{msionization}.
\par Elektron ionisering ioniserer molekylerne ved at skyde dem med elektroner, og virker godt for mange molekyler på gasfase, men giver stor fragmentation \parencite{massspec}.
Denne fragmentation er ikke nødvendigvis en dårlig ting, da det kan gøre identifikationen af molekyler med søgebiblioteker som \textcite{nist23}, bedre.
Det er standart at accelerere elektronerne i MS, der gør brug af EI, til \qty{70}{\eV} for at få standartiserede fragmentationsmønstre, og da det er her mange atomer vil ionisere mest \parencite{massspec}.
\subsection{Elektronkanonen}
\begin{wrapfigure}{O}{0.4\linewidth}
	\centering
	\begin{circuitikz}
		% Electrical
		\draw (0,0.8)  -- (1.2,0.8) to[cute inductor] (1.2,-0.8) to[short,-*] (0.6,-0.8) -- (0,-0.8) to[V,invert] (0,0.8)
		(0.6,-0.8) -- (0.6,-2) to[V=$U_a$,invert] (3.85,-2) -- (3.85,-1.4);
		\node at (1.2,1.2) {Katode};

		% Metal plate
		\draw[fill=gray!20] (4,0.6,0) -- ++(-0.3,0,0) -- ++(0,-2,0) -- ++(0.3,0,0) -- cycle;
		\draw[fill=gray!80] (4,0.6,0) -- ++(0,0,-2) -- ++(0,-2,0) -- ++(0,0,2) -- cycle;
		\draw[fill=gray!50] (4,0.6,0) -- ++(-0.3,0,0) -- ++(0,0,-2) -- ++(0.3,0,0) -- cycle;
		\node at (3.5,1.2) {Anode};

		% Hole
		\begin{scope}[rotate=-90,scale=0.4,shift={(0,10.4375)}]
			\draw[fill=white] (0,0.5) ellipse[x radius=0.75,y radius=0.3];
			\clip (0.75,1) -- (0.75,0) arc[start angle=0,end angle=180,x radius=0.75,y radius=0.3] -- (-0.75,1);
			\clip (0,.5) ellipse[x radius=0.75,y radius=.3];
			\shade[top color=black,bottom color=black,middle color=white] (-1.75,0) rectangle (2,1);
			\draw (0.75,0) arc[start angle=0,end angle=180,x radius=.75,y radius=.3];
			\draw (0,0.5) ellipse[x radius=0.75,y radius=0.3];
		\end{scope}

		% Electron
		\draw[dashed] (1.5,0) -- (3.7,0);
		\draw[-Stealth,red,line width=0.35mm] (2,0) -- (3,0);
		\draw[ball color=blue!60] (2,0) circle (0.1) node[above] {$e^-$};
		\draw[dashed] (4.3,0) -- (5.2,0);
	\end{circuitikz}
	\caption{Simpelt diagram af elektronkanon. Spændingen $U_a$ afgør den kinetiske energi elektronerne forlader kannonen ved.}
	\label{fig:elec-gun}
\end{wrapfigure}

Ved EI beskydes de passerende molekyler af en elektronstråle fra en elektronkanon.
En elektronkanon er i sin mest simple forstand en glødetråd (katoden) og en metalplade med i hul i (anoden)\space\parencite{orbitAstx}.
Glødetråden vil, når den opvarmes, afgive frie elektroner, da disse vil få nok termisk energi til at overstige løsrivelsesarbejdet, der er forskellen mellem Fermi-energien\footnotemark{} og en elektrons energi i vakuum \parencite{electronmicroscopy}.
\footnotetext{Fermi-energien er energien af elektroner i et metal ved det absolutte nulpunkt \parencite{fermienergi}.}
Da der er en potentialeforskel mellem katoden og anoden, vil dette skabe et elektrisk felt, og eftersom feltkraften er det eneste der påvirker elektronerne, vil de blive tiltrukket anoden (se \cref{fig:elec-gun}).
Nogle af elektronerne vil ramme metalpladen, mens andre vil bevæge sig gennem hullet med en kinetisk energi givet ved:
\begin{equation}
	E_{\text{kin}} = e \cdot U_a .
\end{equation}
hvor $e$ er elementarladningen og $U_a$ er spændingsforskellen (se \cref{subsec:e-felt})\space\parencite{grundfysA}.
I et MS med EI, hvor elektronerne accelereres op til at have en kinetisk energi på \qty{70}{\eV} vil det altså kræve en spændingsforskel på \qty{70}{\volt}.
\subsection{Ionisering}
Når de neutralt ladede molekyler i EI'en rammes af en elektron overføres noget af elektronens energi til den.
Hvis elektronen rammer rigtigt kan energien overført være nok til at slå en elektron ud af dens bane, og derved ionisere molekylet til en positiv ion \parencite{mstextbook}.
\begin{reaction}
	M + e- -> M+ + 2 e-
\end{reaction}
EI vil primært producere ioner med ladningen $+1$, men nogle vil få højere iontal, og da dette giver en anden $m/z$ vil de også kunne ses som en anden top i massespektret.
\subsection{Ioniseringsenergi}
\emph{Ioniseringsenergien} (IE) er den minimale energi det vil kræve at ionisere et atom eller et molekyle ved at fjerne en elektron.
Fjernelsen af en elektron fra et molekyle kan ske ved en $\sigma$-binding, en $\pi$-binding eller et elektron lonepair, i faldende grad efter ioniseringsenergi.
Eksempelvis vil ædelgasser med fyldte elektronskaller have en høj IE, mens alkenen ethen vil have lavere IE.
Derudover vil store molekyler have en lavere IE.
De fleste molekyler vil have en ioniseringsenergi omkring \qtyrange{7}{15}{\eV} \parencite{mstextbook}.
