\chapter{Eksperiment}
I forbindelse med problemformuleringen vil der udføres et forsøg, hvor aminosyreforholdet af fire honningprøver, hvoraf to er falske, undersøges.
De fire prøver samt indholdet af dem kan ses i \cref{tab:samples}.
Formålet med forsøget er at se, hvorvidt man ved hjælp af aminosyreforholdende kan afgøre snyd i honning.
Målingerne foretages ved hjælp af en GC-MS\footnotemark{}, som foretager massespektrometriske målinger for hver retentionstid.
\footnotetext{Den benyttede GC-MS var en \href{https://www.gmi-inc.com/product/agilent-hp-5973n-mass-selective-detector/}{Agilent HP 5973N Mass Selective Detector}}
Der vil ikke foretages kalibreringsprøver for at kunne bestemme det nøjagtige indhold i prøverne, men kun det relative indhold vil kunne bestemmes.
Disse målinger vil senere blive benyttet til at undesøge aminosyreforhold for prøven.
\begin{table}[hbt]
	\centering
	\caption{Tabel over blandinger for de fire prøver benyttet i forsøg, som masseprocent, samt den eksakt afmålte mængde brugt i forsøg.}
	\setlength{\extrarowheight}{4pt}
	\begin{tabular}{rlS[table-format=1.3]}\toprule
    P.nr. & Bestandel                                                                                                  & {Eksakt masse $[\unit{\gram}]$} \\\midrule
		1     & \qty{100}{\percent} Dansk honning$^1$                                                                      & 4.973                         \\
		2     & \vtop{\hbox{\strut \qty{75}{\percent} Agavesirup$^2$}\hbox{\strut \qty{25}{\percent} Dansk honning$^1$}}   & 5.048                         \\
		3     & \vtop{\hbox{\strut \qty{75}{\percent} Glukosesirup$^3$}\hbox{\strut \qty{25}{\percent} Dansk honning$^1$}} & 4.954                         \\
		4     & \qty{100}{\percent} Ikke-EU honning$^4$                                                                    & 4.998                         \\\bottomrule
		\multicolumn{3}{l}{
			\footnotesize \vtop{
				\hbox{\strut $^1$\href{https://jakobsens.com/product/dansk-honning-4/}{Jakobsens Dansk flydende Honning}}
				\hbox{\strut $^2$\href{https://jakobsens.com/product/agave-sirup-lys-oekologisk/}{Jakobsens Økologisk lys agave sirup}}
				\hbox{\strut $^3$\href{https://www.dansukker.dk/dk/produkter/alle-produkter/glukosesirup}{Dansukker Glukosesirup}}
				\hbox{\strut $^4$\href{https://jakobsens.com/product/hverdags-honning-3/}{Jakobsens Hverdags flydende honning}}
			}
		}
	\end{tabular}
	\label{tab:samples}
\end{table}

\par Før prøverne var klar til måling skulle der foretages en \emph{Fastfaseekstraktion} (SPE, Solid Phase Extraction) for at ekstrahere aminosyrene fra honningen.
Her udnyttedes aminosyres egenskab med at have forskellig ladning afhængig af pH (se \cref{subsec:aminoPH}).
Derefter blev der foretaget en derivatissering af aminosyrene for at gøre dem mere flygtige.
Nærmere fremgangsmåde samt dokumentation for prøveforberedelse kan findes i \cref{appsec:sample-prep}.
