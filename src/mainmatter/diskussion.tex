\section{MS som analysemetode}
Det lykkedes som sagt at afsløre forfalskningen af de to opblandede honninger ved hjælp af massespektrometri.
Dette skyldtes primært den generelt meget lavere koncentration af aminosyrer i opblandingerne.
Det var dog tydeligt at afgøre de var falske blot ud fra viskositet, farve og lugt, ved de brugte opblandingsforhold.
Altså ville det ved rigtig snyd enten kræve andre opblandingsmidler eller andre opblandingsforhold.
Ved opblandinger hvori der kun blandes lidt sukkersirup, ville det være svært at afgøre forfalskning, da der kan være meget stor forskel på honningers aminosyreindhold, som sås ved sammenligning mellem prøve 1 og 4.
Det ville kræve en større viden om de normale niveauer for aminosyrer i honning afhængig af botanisk og geografisk oprindelse, at kunne sige noget præcist om små opblandinger, som også foreslået i \textcite{geohoney}.
Det ville også have givet mere sammenlignelige resultater, havde kalibreringprøver fundet sted, da den nøjagtige koncentration i prøverne derved ville kunne være blevet bestemt.
\par Honing bliver ofte opblandet med rissirup, fordi det minder mere om honning i sukkerindhold, farve og viskositet end andre sukkersirupper \parencite{BRShoney}.
Her ville det være muligt at bestemme forfalskningen, da aminosyreindholdet er meget forskelligt fra honning og da det ofte opblandes i store forhold \parencite{BRSaa}.
Som tidligere nævnt, er den anden primære måde honning bliver forfalsket på, ved fodring af bierne med billige sukre, oftest sukrose \parencite{adulterationhoney}.
Her vil det også være muligt at afsløre forfalskningen, da de ved indtagelse af sukkersirup ikke vil have hentet nektar fra planter, som er der størstedelen af aminosyrerne kommer fra.
Prolin, som er den primære aminosyre i honning kommer af osmosen foretaget ved optagelsen af nektar \parencite{prolineNectar}.
Begge disse metoder vil resultere i et indhold og fordeling af sukker lig ægte honning, så her kræver det andre indikatorer, som f.eks. aminosyrer, at afsløre snyd.
\par Med massespektrometri vil det også være muligt at afsløre snyd af honning, som ikke har den korrekte geografiske eller botaniske oprindelse angivet.
Dette vil dog også kræve en bedre kortlægning af aminosyreindhold af honning.
\subsection{Analysemetoder}
Sammenkoblingen mellem gaschromatografi og massespektrometri giver let identificering samt god kvantificering af aminosyrerne i honning.
I forsøget blev en firpolet massespektrometer brugt, da den hurtigt kan lave en fuldspektrum scanning.
Dog er denne form for massespektrometri ikke den type med højest præcision eller opløsning, hvilket giver risiko for misidentificering af honning, som vil give forkerte resultater.
Andre former for massespektrometri kaldt \emph{High Resolution} MS (HRMS) indebærer massespektrometre som \emph{Time of flight} MS (TOFMS), \emph{Fourier transform ion cyclotron resonance} MS (FTICR) og Orbitrap MS.
Forskellen på HRMS og traditionelle former for massespektrometri er, at de ikke fragmentere ionerne, men i stedet kan måle deres nøjagtige masse, ofte til et opløsning mindst \num{20} gange bedre en traditionel massespektrometri \parencite{tandemMS}.
De er altså mere præcise, og man mindsker derved chancen for misidentificering. Disse kan også bruges i sammenkobling med chromatografiske metoder \parencite{mstextbook}.
\par Andre metoder til at analysere forfalskning i honning er \emph{Elemental Analyser/Liquid Chromatography - Isotope Ratio Mass Spectrometry} (EA/LS-IRMS),
som undersøger forholdet mellem \isotope*{13,C} og \isotope*{12,C} i proteiner og sukre, High-Performance Anion Exchange Chromatography - Pulsed Amperometric Detector (HPAEC-PAD),
som undersøger polyssaccharider med polymerisation $\geq 10$ og \emph{Proton Nuclear Magnetic Resonance Spectroscopy} (\isotope*{1,H}-NMR), som kan sige noget om næsten hele det kemiske indhold af honning \parencite{EUhoney}.
Disse har alle sine fordele og ulempler i forhold til metoder med massespektrometri, men særligt \isotope*{1,H}-NMR kan bruges til høj præcision og til bestemmelse af både geografisk og botanisk oprindelse.
Her er også oprettet en database med omkring \num{28500} referenceprøver \parencite{HNMRbruker}.
Dog har \isotope*{1,H}-NMR en relativ lav opløsning og er meget dyre i drift, da de skal køles til meget lave temperaturer \parencite{advHNMR}.
\subsection{Vurdering af MS}
Massespektrometri kan godt bruges til at afsløre snyd i honning, såfremt at det ikke kun opblandes lidt.
Dog vil det kræve en bredere viden omkring aminosyreindholdet afhængigt af geografisk og botanisk oprindelse at kunne mindske falsk positive og falsk negative resultater.
Derudover vil det give bedre resultater, hvis en metode med HRMS bliver brugt.
Andre analysemetoder som \isotope*{1,H}-NMR kan også bruges til at afsløre snyd, men kan være meget dyrere i drift og mindre præcise end metoder som gør brug af massespektrometri.
\par Generelt vil det være svært at afgøre med sikkerhed når en honning er forfalsket, da det er et naturprodukt og hver eneste dråbe honning bierne producerer, vil være forskellig.
Metoder kan udvikles til at blive bedre til at opdage snyd i honning, og sikre at forbrugeren får det produkt de betaler for.
